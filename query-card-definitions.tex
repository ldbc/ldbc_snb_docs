% colors from Color Brewer 2.0, Set1: http://colorbrewer2.org/#type=qualitative&scheme=Set1&n=3
\definecolor{parameter}{HTML}{e41a1c}
\definecolor{result}{HTML}{377eb8}
\definecolor{sort}{HTML}{4daf4a}

% using camelCase notation is not conventional in LaTeX, but it helps readability a lot, so I decided to use it anyways [szarnyasg]

\newcommand{\queryCardWidth}{17cm}
\newcommand{\queryPropertyCell}{\small \sf}

\newcommand{\attributeCardWidth}{14.66cm}
\newcommand{\typeWidth}{2cm}

\newcommand{\paramNumberCell}{\cellcolor{parameter} \color{white}\footnotesize}
\newcommand{\resultNumberCell}{\cellcolor{result} \color{white} \footnotesize}
\newcommand{\sortNumberCell}{\cellcolor{sort} \color{white} \footnotesize}

\newcommand{\directionCell}{\cellcolor{gray!20}}
\newcommand{\resultOriginCell}{\tt}

\newcommand{\varName}{\small \tt}
\newcommand{\varNameCell}{\small \normalfont\fontfamily{cmtt}\selectfont}
\newcommand{\typeCell}{\cellcolor{gray!20} \footnotesize \sf \raggedright}

\newcommand{\chokePoint}[1]{\hyperref[choke_point_#1]{#1}}

\newcommand{\innerCardVSpace}{\vspace{1.1ex}}
\newcommand{\queryCardVSpace}{\vspace{2ex}}

% tabularx magic
% https://tex.stackexchange.com/a/89932/71109
\newcolumntype{Y}{>{\raggedright\arraybackslash}X}
% https://tex.stackexchange.com/questions/252385/mixing-m-and-x-in-tabularx#comment602205_252388
\renewcommand{\tabularxcolumn}[1]{m{#1}}
% https://tex.stackexchange.com/a/4712/71109
\newcolumntype{M}{>{\begin{varwidth}{7.8cm}}l<{\end{varwidth}}}
% NOTE. Do note try to use a different value for \arraystretch for nested tables, as it will look bad if the nested table has a single line.
